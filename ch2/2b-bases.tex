\documentclass[11pt]{article}

% =====================================================
% Layout & Typography
% =====================================================
\usepackage[margin=1in]{geometry}
\usepackage{setspace}
\setstretch{1.15}

\usepackage[T1]{fontenc}
\usepackage{lmodern}
\usepackage{microtype}

% =====================================================
% Math
% =====================================================
\usepackage{amsmath, amssymb}
\usepackage{mathtools}

% =====================================================
% Lists & Links
% =====================================================
\usepackage{enumitem}
\usepackage{hyperref}
\hypersetup{
  colorlinks=true,
  linkcolor=blue,
  urlcolor=blue
}

% =====================================================
% Framed Exercise Environment (LEGAL)
% =====================================================
\usepackage{mdframed}

\newcounter{exercise}

\newmdenv[
  linewidth=0.6pt,
  skipabove=1.2em,
  skipbelow=1.2em,
  innerleftmargin=1em,
  innerrightmargin=1em,
  innertopmargin=0.8em,
  innerbottommargin=0.8em
]{exercisebox}

\newenvironment{exercise}[1]
{%
  \refstepcounter{exercise}
  \begin{exercisebox}
  \noindent\textbf{Problem \theexercise.} \textit{#1}\par\medskip
}
{%
  \end{exercisebox}
}

% =====================================================
% Solution Environment
% =====================================================
\newenvironment{solution}
{\par\noindent\textbf{Solution.}\ }
{\hfill$\square$\par}

% =====================================================
% Common Commands
% =====================================================
\newcommand{\R}{\mathbb{R}}
\newcommand{\C}{\mathbb{C}}
\newcommand{\F}{\mathbb{F}}
\newcommand{\Span}{\operatorname{span}}
\newcommand{\Null}{\operatorname{null}}
\newcommand{\Range}{\operatorname{range}}

% =====================================================
% Metadata
% =====================================================
\title{Linear Algebra Done Right \\ \large Section 2A: Span and Linear Independence}
\author{Naman Tyagi}
\date{}

\begin{document}

\maketitle

% =====================================================
% Document
% =====================================================
\begin{exercise}{}
Find all vector spaces that have exactly one basis.
\end{exercise}

\begin{solution}
Let \(V\) be a vector space over a field \(\mathbb{F}\).

If \(\dim V = 0\), then \(V = \{0\}\). The empty set is a basis of \(V\), and it is the only basis.

Now suppose \(\dim V \ge 1\). Let \(\{v_1,\dots,v_n\}\) be a basis of \(V\). For any scalar \(\lambda \in \mathbb{F}\) with \(\lambda \neq 0,1\), the set
\[
\{\lambda v_1, v_2,\dots,v_n\}
\]
is also a basis of \(V\), distinct from the original one. Hence \(V\) has more than one basis.

Therefore, the only vector space with exactly one basis is the zero vector space.
\end{solution}


\begin{exercise}{}
Verify all assertions in Example 2.27.
\end{exercise}

\begin{solution}
(a) Let \(e_1,\dots,e_n\) denote the vectors
\[
(1,0,\dots,0),\ (0,1,0,\dots,0),\ \dots,\ (0,\dots,0,1)
\]
in \(\mathbb{F}^n\). Every vector \((x_1,\dots,x_n)\in\mathbb{F}^n\) can be written as
\[
x_1 e_1 + \cdots + x_n e_n,
\]
so the list spans \(\mathbb{F}^n\). Linear independence is immediate because a linear combination
\[
a_1 e_1 + \cdots + a_n e_n = 0
\]
implies \(a_1=\cdots=a_n=0\). Hence this list is a basis of \(\mathbb{F}^n\).

(b) Consider \((1,2),(3,5)\in\mathbb{F}^2\). If
\[
a(1,2)+b(3,5)=(0,0),
\]
then
\[
a+3b=0,\qquad 2a+5b=0.
\]
Solving gives \(a=b=0\), so the list is linearly independent. Because it has length \(2=\dim\mathbb{F}^2\), it is a basis of \(\mathbb{F}^2\).

(c) Consider \((1,2,-4),(7,-5,6)\in\mathbb{F}^3\). If
\[
a(1,2,-4)+b(7,-5,6)=0,
\]
then
\[
a+7b=0,\quad 2a-5b=0,\quad -4a+6b=0.
\]
The first two equations imply \(a=b=0\), so the list is linearly independent. However, it has length \(2<3\), so it cannot span \(\mathbb{F}^3\) and hence is not a basis.

(d) The list \((1,2),(3,5),(4,13)\) spans \(\mathbb{F}^2\) because \((1,2),(3,5)\) already form a basis. However, any list of three vectors in \(\mathbb{F}^2\) is linearly dependent. Thus this list is not a basis.

(e) Let
\[
U=\{(x,x,y)\in\mathbb{F}^3:x,y\in\mathbb{F}\}.
\]
Every element of \(U\) can be written as
\[
(x,x,y)=x(1,1,0)+y(0,0,1),
\]
so the list \((1,1,0),(0,0,1)\) spans \(U\). If
\[
a(1,1,0)+b(0,0,1)=0,
\]
then \((a,a,b)=(0,0,0)\), implying \(a=b=0\). Thus the list is linearly independent and hence a basis of \(U\).

(f) Let
\[
W=\{(x,y,z)\in\mathbb{F}^3:x+y+z=0\}.
\]
For \(x,y\in\mathbb{F}\),
\[
(x,y,-x-y)=x(1,0,-1)+y(0,1,-1)
= x(1,-1,0)+ (x+y)(1,0,-1),
\]
so \(W\) is spanned by \((1,-1,0)\) and \((1,0,-1)\). If
\[
a(1,-1,0)+b(1,0,-1)=0,
\]
then \((a+b,-a,-b)=(0,0,0)\), which implies \(a=b=0\). Hence the list is a basis of \(W\).

(g) Let \(\mathcal{P}_m(\mathbb{F})\) denote the vector space of polynomials over \(\mathbb{F}\) of degree at most \(m\). Every \(p\in\mathcal{P}_m(\mathbb{F})\) can be written uniquely as
\[
p(z)=a_0+a_1 z+\cdots+a_m z^m,
\]
so the list \(1,z,\dots,z^m\) spans \(\mathcal{P}_m(\mathbb{F})\). If
\[
a_0+a_1 z+\cdots+a_m z^m=0,
\]
then all coefficients are zero, so the list is linearly independent. Therefore it is a basis of \(\mathcal{P}_m(\mathbb{F})\).
\end{solution}

\begin{exercise}{}
(a) Let \(U\) be the subspace of \(\mathbb{R}^5\) defined by
\[
U=\{(x_1,x_2,x_3,x_4,x_5)\in\mathbb{R}^5 : x_1=3x_2 \text{ and } x_3=7x_4\}.
\]
Find a basis of \(U\).

(b) Extend the basis in (a) to a basis of \(\mathbb{R}^5\).

(c) Find a subspace \(W\) of \(\mathbb{R}^5\) such that \(\mathbb{R}^5 = U \oplus W\).
\end{exercise}

\begin{solution}
(a) Any vector in \(U\) has the form
\[
(x_1,x_2,x_3,x_4,x_5)=(3x_2,x_2,7x_4,x_4,x_5)
= x_2(3,1,0,0,0)+x_4(0,0,7,1,0)+x_5(0,0,0,0,1).
\]
Thus
\[
\{(3,1,0,0,0),(0,0,7,1,0),(0,0,0,0,1)\}
\]
spans \(U\). It is linearly independent, so it is a basis of \(U\).

(b) To extend this to a basis of \(\mathbb{R}^5\), add vectors that are linearly independent from those in (a), for example \((1,0,0,0,0)\) and \((0,0,1,0,0)\). Then
\[
\{(3,1,0,0,0),(0,0,7,1,0),(0,0,0,0,1),(1,0,0,0,0),(0,0,1,0,0)\}
\]
is a basis of \(\mathbb{R}^5\).

(c) Let
\[
W=\operatorname{span}\{(1,0,0,0,0),(0,0,1,0,0)\}.
\]
Then \(U\cap W=\{0\}\) and \(\dim U+\dim W=3+2=5=\dim\mathbb{R}^5\). Hence
\[
\mathbb{R}^5=U\oplus W.
\]
\end{solution}

\begin{exercise}{}
(a) Let \(U\) be the subspace of \(\mathbb{C}^5\) defined by
\[
U=\{(z_1,z_2,z_3,z_4,z_5)\in\mathbb{C}^5 : 6z_1=z_2 \text{ and } z_3+2z_4+3z_5=0\}.
\]
Find a basis of \(U\).

(b) Extend the basis in (a) to a basis of \(\mathbb{C}^5\).

(c) Find a subspace \(W\) of \(\mathbb{C}^5\) such that \(\mathbb{C}^5 = U \oplus W\).
\end{exercise}

\begin{solution}
(a) Any vector in \(U\) has the form
\[
(z_1,6z_1,z_3,z_4,z_5)
\quad\text{with}\quad z_3=-2z_4-3z_5.
\]
Thus
\[
(z_1,6z_1,-2z_4-3z_5,z_4,z_5)
= z_1(1,6,0,0,0)+z_4(0,0,-2,1,0)+z_5(0,0,-3,0,1).
\]
Hence
\[
\{(1,6,0,0,0),(0,0,-2,1,0),(0,0,-3,0,1)\}
\]
is a basis of \(U\).

(b) Add two vectors linearly independent from those in (a), for example \((0,1,0,0,0)\) and \((0,0,1,0,0)\). Then
\[
\{(1,6,0,0,0),(0,0,-2,1,0),(0,0,-3,0,1),(0,1,0,0,0),(0,0,1,0,0)\}
\]
is a basis of \(\mathbb{C}^5\).

(c) Let
\[
W=\operatorname{span}\{(0,1,0,0,0),(0,0,1,0,0)\}.
\]
Then \(U\cap W=\{0\}\) and \(\dim U+\dim W=3+2=5\). Hence
\[
\mathbb{C}^5=U\oplus W.
\]
\end{solution}

\begin{exercise}{}
Suppose \(V\) is finite-dimensional and \(U, W\) are subspaces of \(V\) such that
\(V = U + W\). Prove that there exists a basis of \(V\) consisting of vectors in
\(U \cup W\).
\end{exercise}

\begin{solution}
Let \(\{u_1,\dots,u_m\}\) be a basis of \(U\). Let \(\{w_1,\dots,w_k\}\) be a basis of \(W\).

The list
\[
u_1,\dots,u_m,w_1,\dots,w_k
\]
spans \(V\) because \(V = U + W\). Remove vectors from this list one at a time until a basis of \(V\) remains. Each remaining vector is still one of the \(u_i\) or one of the \(w_j\).

Thus we obtain a basis of \(V\) consisting entirely of vectors in \(U \cup W\).
\end{solution}

\begin{exercise}{}
Prove or give a counterexample: If \(p_0,p_1,p_2,p_3\) is a list in \(\mathcal{P}_3(\mathbb{F})\) such that none of the polynomials has degree \(2\), then \(p_0,p_1,p_2,p_3\) is not a basis of \(\mathcal{P}_3(\mathbb{F})\).
\end{exercise}

\begin{solution}
The statement is false.

Consider the list
\[
1,\; z,\; z^3,\; z^3+z^2
\]
in \(\mathcal{P}_3(\mathbb{F})\). None of these polynomials has degree \(2\).

The list is linearly independent because the coefficient vectors with respect to
\(\{1,z,z^2,z^3\}\) are linearly independent. Moreover, the polynomial \(z^2\) can be written as
\[
z^2 = (z^3+z^2) - z^3,
\]
so the list spans \(\mathcal{P}_3(\mathbb{F})\).

Hence this list is a basis of \(\mathcal{P}_3(\mathbb{F})\), even though none of its elements has degree \(2\).
\end{solution}

\begin{exercise}{}
Suppose \(v_1,v_2,v_3,v_4\) is a basis of \(V\). Prove that
\[
v_1+v_2,\; v_2+v_3,\; v_3+v_4,\; v_4
\]
is also a basis of \(V\).
\end{exercise}

\begin{solution}
Because \(v_1,v_2,v_3,v_4\) is a basis of \(V\), it suffices to show that the given list is linearly independent.

Suppose
\[
a_1(v_1+v_2)+a_2(v_2+v_3)+a_3(v_3+v_4)+a_4 v_4=0.
\]
Rewriting,
\[
a_1 v_1+(a_1+a_2)v_2+(a_2+a_3)v_3+(a_3+a_4)v_4=0.
\]
Since \(v_1,v_2,v_3,v_4\) is linearly independent, all coefficients must be zero:
\[
a_1=0,\quad a_1+a_2=0,\quad a_2+a_3=0,\quad a_3+a_4=0.
\]
From \(a_1=0\) we get \(a_2=0\), then \(a_3=0\), and finally \(a_4=0\).

Thus the list is linearly independent. Because it has four vectors in a four-dimensional space, it is a basis of \(V\).
\end{solution}

\begin{exercise}{}
Prove or give a counterexample: If \(v_1,v_2,v_3,v_4\) is a basis of \(V\) and \(U\) is a
subspace of \(V\) such that \(v_1,v_2 \in U\) and \(v_3 \notin U\) and \(v_4 \notin U\),
then \(v_1,v_2\) is a basis of \(U\).
\end{exercise}

\begin{solution}
The statement is false.

Let \(V=\mathbb{F}^4\) and let
\[
v_1=e_1,\quad v_2=e_2,\quad v_3=e_3,\quad v_4=e_4,
\]
the standard basis of \(\mathbb{F}^4\). Define
\[
U=\operatorname{span}\{e_1,e_2,e_3+e_4\}.
\]
Then \(v_1,v_2\in U\), but \(v_3\notin U\) and \(v_4\notin U\).

However, \(\dim U=3\), so the list \(v_1,v_2\) does not span \(U\) and hence is not a basis of \(U\).
\end{solution}

\begin{exercise}{}
Suppose \(v_1,\dots,v_m\) is a list of vectors in \(V\). For \(k\in\{1,\dots,m\}\), let
\[
w_k = v_1+\cdots+v_k.
\]
Show that \(v_1,\dots,v_m\) is a basis of \(V\) if and only if \(w_1,\dots,w_m\) is a
basis of \(V\).
\end{exercise}

\begin{solution}
Note that
\[
w_1=v_1 \quad\text{and}\quad w_k-w_{k-1}=v_k \ \text{for } k=2,\dots,m.
\]
Thus each \(w_k\) lies in \(\operatorname{span}(v_1,\dots,v_m)\) and each \(v_k\) lies in
\(\operatorname{span}(w_1,\dots,w_m)\). Hence
\[
\operatorname{span}(v_1,\dots,v_m)=\operatorname{span}(w_1,\dots,w_m).
\]

If \(v_1,\dots,v_m\) is a basis of \(V\), then it spans \(V\) and is linearly independent.
The equality of spans implies that \(w_1,\dots,w_m\) spans \(V\). To show linear
independence, suppose
\[
a_1 w_1+\cdots+a_m w_m=0.
\]
Rewriting,
\[
(a_1+\cdots+a_m)v_1+(a_2+\cdots+a_m)v_2+\cdots+a_m v_m=0.
\]
Because \(v_1,\dots,v_m\) is linearly independent, we obtain \(a_m=0\), then
\(a_{m-1}=0\), and continuing, \(a_1=0\). Hence \(w_1,\dots,w_m\) is linearly
independent and therefore a basis of \(V\).

Conversely, if \(w_1,\dots,w_m\) is a basis of \(V\), then it spans \(V\) and is linearly
independent. The equality of spans implies that \(v_1,\dots,v_m\) spans \(V\). If
\[
b_1 v_1+\cdots+b_m v_m=0,
\]
then writing each \(v_k\) in terms of the \(w_j\) gives a linear combination of
\(w_1,\dots,w_m\) equal to \(0\). By linear independence of the \(w_j\), all
coefficients are zero, so \(b_1=\cdots=b_m=0\). Thus \(v_1,\dots,v_m\) is linearly
independent and hence a basis of \(V\).
\end{solution}

\begin{exercise}{}
Suppose \(v_1,\dots,v_m\) is a list of vectors in \(V\). For \(k\in\{1,\dots,m\}\), let
\[
w_k = v_1+\cdots+v_k.
\]
Show that \(v_1,\dots,v_m\) is a basis of \(V\) if and only if \(w_1,\dots,w_m\) is a
basis of \(V\).
\end{exercise}

\begin{solution}
Note that
\[
w_1=v_1 \quad\text{and}\quad w_k-w_{k-1}=v_k \ \text{for } k=2,\dots,m.
\]
Thus each \(w_k\) lies in \(\operatorname{span}(v_1,\dots,v_m)\) and each \(v_k\) lies in
\(\operatorname{span}(w_1,\dots,w_m)\). Hence
\[
\operatorname{span}(v_1,\dots,v_m)=\operatorname{span}(w_1,\dots,w_m).
\]

If \(v_1,\dots,v_m\) is a basis of \(V\), then it spans \(V\) and is linearly independent.
The equality of spans implies that \(w_1,\dots,w_m\) spans \(V\). To show linear
independence, suppose
\[
a_1 w_1+\cdots+a_m w_m=0.
\]
Rewriting,
\[
(a_1+\cdots+a_m)v_1+(a_2+\cdots+a_m)v_2+\cdots+a_m v_m=0.
\]
Because \(v_1,\dots,v_m\) is linearly independent, we obtain \(a_m=0\), then
\(a_{m-1}=0\), and continuing, \(a_1=0\). Hence \(w_1,\dots,w_m\) is linearly
independent and therefore a basis of \(V\).

Conversely, if \(w_1,\dots,w_m\) is a basis of \(V\), then it spans \(V\) and is linearly
independent. The equality of spans implies that \(v_1,\dots,v_m\) spans \(V\). If
\[
b_1 v_1+\cdots+b_m v_m=0,
\]
then writing each \(v_k\) in terms of the \(w_j\) gives a linear combination of
\(w_1,\dots,w_m\) equal to \(0\). By linear independence of the \(w_j\), all
coefficients are zero, so \(b_1=\cdots=b_m=0\). Thus \(v_1,\dots,v_m\) is linearly
independent and hence a basis of \(V\).
\end{solution}

\begin{exercise}{}
Suppose \(V\) is a real vector space. Show that if \(v_1,\dots,v_n\) is a basis of \(V\)
(as a real vector space), then \(v_1,\dots,v_n\) is also a basis of the
complexification \(V_{\mathbb{C}}\) (as a complex vector space).
\end{exercise}

\begin{solution}
Recall that \(V_{\mathbb{C}}=\{u+iv : u,v\in V\}\), with complex scalar multiplication.

First we show that \(v_1,\dots,v_n\) spans \(V_{\mathbb{C}}\).
Let \(u+iv\in V_{\mathbb{C}}\). Because \(v_1,\dots,v_n\) is a real basis of \(V\),
there exist real scalars \(a_1,\dots,a_n\) and \(b_1,\dots,b_n\) such that
\[
u=\sum_{k=1}^n a_k v_k
\quad\text{and}\quad
v=\sum_{k=1}^n b_k v_k.
\]
Then
\[
u+iv=\sum_{k=1}^n (a_k+ib_k)v_k,
\]
so \(v_1,\dots,v_n\) spans \(V_{\mathbb{C}}\) over \(\mathbb{C}\).

Next we show linear independence. Suppose
\[
c_1 v_1+\cdots+c_n v_n=0
\]
with \(c_k\in\mathbb{C}\). Write \(c_k=a_k+ib_k\) with \(a_k,b_k\in\mathbb{R}\). Then
\[
\sum_{k=1}^n a_k v_k + i\sum_{k=1}^n b_k v_k = 0.
\]
Thus
\[
\sum_{k=1}^n a_k v_k = 0
\quad\text{and}\quad
\sum_{k=1}^n b_k v_k = 0.
\]
Because \(v_1,\dots,v_n\) is linearly independent over \(\mathbb{R}\), we obtain
\(a_k=b_k=0\) for all \(k\). Hence \(c_k=0\) for all \(k\), so the list is linearly
independent over \(\mathbb{C}\).

Therefore \(v_1,\dots,v_n\) is a basis of \(V_{\mathbb{C}}\).
\end{solution}

\end{document}
