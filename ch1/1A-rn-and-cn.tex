\documentclass[11pt]{article}

% =====================================================
% Layout & Typography
% =====================================================
\usepackage[margin=1in]{geometry}
\usepackage{setspace}
\setstretch{1.15}

\usepackage[T1]{fontenc}
\usepackage{lmodern}
\usepackage{microtype}

% =====================================================
% Math
% =====================================================
\usepackage{amsmath, amssymb}
\usepackage{mathtools}

% =====================================================
% Lists & Links
% =====================================================
\usepackage{enumitem}
\usepackage{hyperref}
\hypersetup{
  colorlinks=true,
  linkcolor=blue,
  urlcolor=blue
}

% =====================================================
% Framed Exercise Environment (LEGAL)
% =====================================================
\usepackage{mdframed}

\newcounter{exercise}

\newmdenv[
  linewidth=0.6pt,
  skipabove=1.2em,
  skipbelow=1.2em,
  innerleftmargin=1em,
  innerrightmargin=1em,
  innertopmargin=0.8em,
  innerbottommargin=0.8em
]{exercisebox}

\newenvironment{exercise}[1]
{%
  \refstepcounter{exercise}
  \begin{exercisebox}
  \noindent\textbf{Problem \theexercise.} \textit{#1}\par\medskip
}
{%
  \end{exercisebox}
}

% =====================================================
% Solution Environment
% =====================================================
\newenvironment{solution}
{\par\noindent\textbf{Solution.}\ }
{\hfill$\square$\par}

% =====================================================
% Common Commands
% =====================================================
\newcommand{\R}{\mathbb{R}}
\newcommand{\C}{\mathbb{C}}
\newcommand{\F}{\mathbb{F}}
\newcommand{\Span}{\operatorname{span}}
\newcommand{\Null}{\operatorname{null}}
\newcommand{\Range}{\operatorname{range}}

% =====================================================
% Metadata
% =====================================================
\title{Linear Algebra Done Right \\ \large Section 1A: $\R^n$ and $\C^n$}
\author{Naman Tyagi}
\date{}

% =====================================================
% Document
% =====================================================
\begin{document}
\maketitle

\begin{exercise}{}
Show that $\alpha + \beta = \beta + \alpha\quad   \forall \alpha, \beta \in \C$.
\end{exercise}

\begin{solution}
We know that we can write $\alpha$ and $\beta$ as 
\[
\begin{aligned}
\alpha &= a_1 + ib_1, \\
\beta  &= a_2 + ib_2 .
\end{aligned}
\]

Since addition in $\mathbb{R}$ is commutative, we may rearrange terms. Thus,
\[
\begin{aligned}
\alpha + \beta &= (a_1 + ib_1) + (a_2 + ib_2) \\
&= a_1 + a_2 + i(b_1 + b_2) \\
&= a_2 + a_1 + i(b_2 + b_1) \\
&= (a_2 + ib_2) + (a_1 + ib_1) \\
&= \beta + \alpha .
\end{aligned}
\]


\end{solution}

\begin{exercise}{}
Show that $(\alpha + \beta) + \lambda = \alpha + (\beta + \lambda)\quad
\forall \alpha, \beta, \lambda \in \mathbb{C}$.
\end{exercise}

\begin{solution}
Let
\[
\alpha = a_1 + b_1 i,\quad
\beta = a_2 + b_2 i,\quad
\lambda = a_3 + b_3 i,
\]
where $a_n, b_n \in \mathbb{R}$.

Then
\[
\begin{aligned}
(\alpha + \beta) + \lambda
&= (a_1 + a_2 + a_3) + (b_1 + b_2 + b_3)i \\
&= a_1 + b_1 i + (a_2 + a_3) + (b_2 + b_3)i \\
&= \alpha + (\beta + \lambda),
\end{aligned}
\]
since addition in $\mathbb{R}$ is associative.
\end{solution}

\begin{exercise}{}
Show that $(\alpha\beta)\lambda = \alpha(\beta\lambda)\quad
\forall \alpha, \beta, \lambda \in \mathbb{C}$.
\end{exercise}

\begin{solution}
Let
\[
\alpha = a_1 + b_1 i,\quad
\beta = a_2 + b_2 i,\quad
\lambda = a_3 + b_3 i,
\]
where $a_n, b_n \in \mathbb{R}$.

First compute $\alpha\beta$:
\[
\begin{aligned}
\alpha\beta
&= (a_1 + b_1 i)(a_2 + b_2 i) \\
&= (a_1a_2 - b_1b_2) + (a_1b_2 + a_2b_1)i.
\end{aligned}
\]

Now,
\[
\begin{aligned}
(\alpha\beta)\lambda
&= \bigl((a_1a_2 - b_1b_2) + (a_1b_2 + a_2b_1)i\bigr)(a_3 + b_3 i) \\
&= (a_1a_2a_3 - b_1b_2a_3 - a_1b_2b_3 - a_2b_1b_3) \\
&\quad + (a_1a_2b_3 - b_1b_2b_3 + a_1b_2a_3 + a_2b_1a_3)i.
\end{aligned}
\]

Next compute $\beta\lambda$:
\[
\begin{aligned}
\beta\lambda
&= (a_2 + b_2 i)(a_3 + b_3 i) \\
&= (a_2a_3 - b_2b_3) + (a_2b_3 + a_3b_2)i.
\end{aligned}
\]

Then,
\[
\begin{aligned}
\alpha(\beta\lambda)
&= (a_1 + b_1 i)\bigl((a_2a_3 - b_2b_3) + (a_2b_3 + a_3b_2)i\bigr) \\
&= (a_1a_2a_3 - a_1b_2b_3 - b_1a_2b_3 - b_1a_3b_2) \\
&\quad + (a_1a_2b_3 + a_1a_3b_2 - b_1b_2b_3 + b_1a_2a_3)i.
\end{aligned}
\]

Reordering terms using associativity and commutativity of addition and
multiplication in $\mathbb{R}$, we see that
\[
(\alpha\beta)\lambda = \alpha(\beta\lambda).
\]
\end{solution}

\begin{exercise}
Show that for every $\alpha \in \mathbb{C}$, there exists a unique
$\beta \in \mathbb{C}$ such that $\alpha + \beta = 0$.
\end{exercise}

\begin{solution}
Let $\alpha \in \mathbb{C}$. Then $\alpha$ can be written as
\[
\alpha = a + bi,
\]
where $a, b \in \mathbb{R}$.

Define
\[
\beta = -a - bi.
\]
Then $\beta \in \mathbb{C}$ and
\[
\alpha + \beta = (a + bi) + (-a - bi) = 0.
\]
Hence, such a $\beta$ exists.

Now suppose that $\beta_1, \beta_2 \in \mathbb{C}$ both satisfy
\[
\alpha + \beta_1 = 0 \quad \text{and} \quad \alpha + \beta_2 = 0.
\]
Then
\[
\alpha + \beta_1 = \alpha + \beta_2.
\]
By adding $-\alpha$ to both sides, we obtain
\[
\beta_1 = \beta_2.
\]
Therefore, $\beta$ is unique.
\end{solution}

\begin{exercise}{}
Show that for every $\alpha \in \mathbb{C}$ with $\alpha \neq 0$,
there exists a unique $\beta \in \mathbb{C}$ such that
$\alpha\beta = 1$.
\end{exercise}

\begin{solution}
Let $\alpha \in \mathbb{C}$ with $\alpha \neq 0$. Then $\alpha$ can be written as
\[
\alpha = a + bi,
\]
where $a, b \in \mathbb{R}$ and not both $a$ and $b$ are zero.

Define
\[
\beta = \frac{a - bi}{a^2 + b^2}.
\]
Since $a^2 + b^2 \neq 0$, $\beta \in \mathbb{C}$.

Now compute
\[
\begin{aligned}
\alpha\beta
&= (a + bi)\frac{a - bi}{a^2 + b^2} \\
&= \frac{a^2 + b^2}{a^2 + b^2} \\
&= 1.
\end{aligned}
\]
Hence, such a $\beta$ exists.

Now suppose that $\beta_1, \beta_2 \in \mathbb{C}$ both satisfy
\[
\alpha\beta_1 = 1 \quad \text{and} \quad \alpha\beta_2 = 1.
\]
Then
\[
\alpha\beta_1 = \alpha\beta_2.
\]
Since $\alpha \neq 0$, multiplying both sides by $\alpha^{-1}$ gives
\[
\beta_1 = \beta_2.
\]
Therefore, $\beta$ is unique.
\end{solution}

\begin{exercise}{}
Show that
\[
\frac{-1 + \sqrt{3}i}{2}
\]
is a cube root of $1$.
\end{exercise}

\begin{solution}
Let
\[
z = \frac{-1 + \sqrt{3}i}{2}.
\]

First compute $z^2$:
\[
\begin{aligned}
z^2
&= \left(\frac{-1 + \sqrt{3}i}{2}\right)^2 \\
&= \frac{1 - 2\sqrt{3}i + 3i^2}{4} \\
&= \frac{-2 - 2\sqrt{3}i}{4} \\
&= \frac{-1 - \sqrt{3}i}{2}.
\end{aligned}
\]

Now compute $z^3$:
\[
\begin{aligned}
z^3
&= z^2 \cdot z \\
&= \left(\frac{-1 - \sqrt{3}i}{2}\right)
   \left(\frac{-1 + \sqrt{3}i}{2}\right) \\
&= \frac{(-1)^2 - (\sqrt{3}i)^2}{4} \\
&= \frac{1 - (-3)}{4} \\
&= 1.
\end{aligned}
\]

Hence,
\[
\left(\frac{-1 + \sqrt{3}i}{2}\right)^3 = 1,
\]
and therefore $\frac{-1 + \sqrt{3}i}{2}$ is a cube root of unity.
\end{solution}

\begin{solution}[Alternatively]
Let
\[
z = \frac{-1 + \sqrt{3}i}{2}.
\]
Note that
\[
\Re(z) = -\frac{1}{2}, \qquad \Im(z) = \frac{\sqrt{3}}{2}.
\]
Thus,
\[
z = \cos\!\left(\frac{2\pi}{3}\right)
+ i\sin\!\left(\frac{2\pi}{3}\right).
\]

By De Moivre's theorem, for any integer $n$,
\[
\left(\cos\theta + i\sin\theta\right)^n
= \cos(n\theta) + i\sin(n\theta).
\]
Applying this with $\theta = \frac{2\pi}{3}$ and $n = 3$, we obtain
\[
\begin{aligned}
z^3
&= \cos(2\pi) + i\sin(2\pi) \\
&= 1.
\end{aligned}
\]

Hence, $\frac{-1 + \sqrt{3}i}{2}$ is a cube root of $1$.
\end{solution}

\begin{exercise}{}
Find two distinct square roots of $i$.
\end{exercise}

\begin{solution}
Let
\[
x = a + bi,
\]
where $a, b \in \mathbb{R}$. Then
\[
\begin{aligned}
x^2
&= (a + bi)^2 \\
&= a^2 - b^2 + 2abi.
\end{aligned}
\]

We require $x^2 = i = 0 + 1i$. Equating real and imaginary parts gives
\[
\begin{cases}
a^2 - b^2 = 0, \\
2ab = 1.
\end{cases}
\]

From $a^2 - b^2 = 0$, we have $a^2 = b^2$, hence $a = \pm b$.

Substituting into $2ab = 1$, we obtain
\[
2a^2 = 1,
\]
so
\[
a^2 = \frac{1}{2}.
\]
Thus
\[
a = \pm \frac{1}{\sqrt{2}},
\quad
b = \pm \frac{1}{\sqrt{2}},
\]
with the same sign as $a$.

Hence, the two distinct square roots of $i$ are
\[
\frac{1}{\sqrt{2}}(1 + i)
\quad \text{and} \quad
-\frac{1}{\sqrt{2}}(1 + i).
\]
\end{solution}

\begin{exercise}{}
Find $x \in \mathbb{R}^4$ such that
\[
(4, -3, 1, 7) + 2x = (5, 9, -6, 8).
\]
\end{exercise}

\begin{solution}
Let
\[
x = (x_1, x_2, x_3, x_4) \in \mathbb{R}^4.
\]
Then
\[
(4, -3, 1, 7) + 2x
= (4 + 2x_1,\, -3 + 2x_2,\, 1 + 2x_3,\, 7 + 2x_4).
\]

Equating components with $(5, 9, -6, 8)$, we obtain
\[
\begin{cases}
4 + 2x_1 = 5, \\
-3 + 2x_2 = 9, \\
1 + 2x_3 = -6, \\
7 + 2x_4 = 8.
\end{cases}
\]

Solving,
\[
x_1 = \frac{1}{2}, \quad
x_2 = 6, \quad
x_3 = -\frac{7}{2}, \quad
x_4 = \frac{1}{2}.
\]

Hence,
\[
x = \left(\frac{1}{2},\, 6,\, -\frac{7}{2},\, \frac{1}{2}\right).
\]
\end{solution}

\begin{exercise}{}
Explain why there does not exist $\lambda \in \mathbb{C}$ such that
\[
\lambda(2 - 3i,\, 5 + 4i,\, -6 + 7i)
= (12 - 5i,\, 7 + 22i,\, -32 - 9i).
\]
\end{exercise}

\begin{solution}
Suppose, for the sake of contradiction, that there exists
$\lambda \in \mathbb{C}$ such that
\[
\lambda(2 - 3i,\, 5 + 4i,\, -6 + 7i)
= (12 - 5i,\, 7 + 22i,\, -32 - 9i).
\]

Then equality of vectors in $\mathbb{C}^3$ implies equality of
corresponding components, so
\[
\begin{cases}
\lambda(2 - 3i) = 12 - 5i, \\
\lambda(5 + 4i) = 7 + 22i, \\
\lambda(-6 + 7i) = -32 - 9i.
\end{cases}
\]

From the first equation,
\[
\lambda = \dfrac{12 - 5i}{2 - 3i}.
\]
From the second equation,
\[
\lambda = \dfrac{7 + 22i}{5 + 4i}.
\]

Since
\[
\dfrac{12 - 5i}{2 - 3i} \neq \dfrac{7 + 22i}{5 + 4i},
\]
no single complex number $\lambda$ can satisfy both equations.
Hence, there does not exist $\lambda \in \mathbb{C}$ such that
\[
\lambda(2 - 3i,\, 5 + 4i,\, -6 + 7i)
= (12 - 5i,\, 7 + 22i,\, -32 - 9i).
\]
\end{solution}

\begin{exercise}{}
Show that $(x + y) + z = x + (y + z)$ for all $x, y, z \in \mathbb{F}^n$.
\end{exercise}

\begin{solution}
Let
\[
x = (x_1,\dots,x_n),\quad
y = (y_1,\dots,y_n),\quad
z = (z_1,\dots,z_n) \in \mathbb{F}^n.
\]
Then
\[
(x+y)+z
= (x_1+y_1+z_1,\dots,x_n+y_n+z_n)
= x+(y+z),
\]
since addition in $\mathbb{F}$ is associative.
\end{solution}

\begin{exercise}{}
Show that $(ab)x = a(bx)$ for all $x \in \mathbb{F}^n$ and all $a,b \in \mathbb{F}$.
\end{exercise}

\begin{solution}
Let $x = (x_1,\dots,x_n) \in \mathbb{F}^n$. Then
\[
(ab)x = (abx_1,\dots,abx_n)
= a(bx_1,\dots,bx_n)
= a(bx),
\]
since multiplication in $\mathbb{F}$ is associative.
\end{solution}

\begin{exercise}{}
Show that $1x = x$ for all $x \in \mathbb{F}^n$.
\end{exercise}

\begin{solution}
Let $x = (x_1,\dots,x_n) \in \mathbb{F}^n$. Then
\[
1x = (1x_1,\dots,1x_n) = (x_1,\dots,x_n) = x,
\]
since $1$ is the multiplicative identity in $\mathbb{F}$.
\end{solution}

\begin{exercise}{}
Show that $\lambda(x+y) = \lambda x + \lambda y$
for all $\lambda \in \mathbb{F}$ and all $x,y \in \mathbb{F}^n$.
\end{exercise}

\begin{solution}
Let
\[
x = (x_1,\dots,x_n),\quad
y = (y_1,\dots,y_n) \in \mathbb{F}^n.
\]
Then
\[
\lambda(x+y)
= (\lambda(x_1+y_1),\dots,\lambda(x_n+y_n))
= (\lambda x_1+\lambda y_1,\dots,\lambda x_n+\lambda y_n)
= \lambda x + \lambda y,
\]
since multiplication distributes over addition in $\mathbb{F}$.
\end{solution}

\begin{exercise}{}
Show that $(a+b)x = ax + bx$
for all $a,b \in \mathbb{F}$ and all $x \in \mathbb{F}^n$.
\end{exercise}

\begin{solution}
Let $x = (x_1,\dots,x_n) \in \mathbb{F}^n$. Then
\[
(a+b)x
= ((a+b)x_1,\dots,(a+b)x_n)
= (ax_1+bx_1,\dots,ax_n+bx_n)
= ax + bx,
\]
since multiplication distributes over addition in $\mathbb{F}$.
\end{solution}

\begin{exercise}{}
Show that $(a+b)x = ax + bx$
for all $a,b \in \mathbb{F}$ and all $x \in \mathbb{F}^n$.
\end{exercise}

\begin{solution}
Let $x = (x_1,\dots,x_n) \in \mathbb{F}^n$. Then
\[
(a+b)x
= ((a+b)x_1,\dots,(a+b)x_n)
= (ax_1+bx_1,\dots,ax_n+bx_n)
= ax + bx,
\]
since multiplication distributes over addition in $\mathbb{F}$.
\end{solution}

\end{document}

