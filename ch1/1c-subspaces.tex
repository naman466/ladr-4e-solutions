\documentclass[11pt]{article}

% =====================================================
% Layout & Typography
% =====================================================
\usepackage[margin=1in]{geometry}
\usepackage{setspace}
\setstretch{1.15}

\usepackage[T1]{fontenc}
\usepackage{lmodern}
\usepackage{microtype}

% =====================================================
% Math
% =====================================================
\usepackage{amsmath, amssymb}
\usepackage{mathtools}

% =====================================================
% Lists & Links
% =====================================================
\usepackage{enumitem}
\usepackage{hyperref}
\hypersetup{
  colorlinks=true,
  linkcolor=blue,
  urlcolor=blue
}

% =====================================================
% Framed Exercise Environment (LEGAL)
% =====================================================
\usepackage{mdframed}

\newcounter{exercise}

\newmdenv[
  linewidth=0.6pt,
  skipabove=1.2em,
  skipbelow=1.2em,
  innerleftmargin=1em,
  innerrightmargin=1em,
  innertopmargin=0.8em,
  innerbottommargin=0.8em
]{exercisebox}

\newenvironment{exercise}[1]
{%
  \refstepcounter{exercise}
  \begin{exercisebox}
  \noindent\textbf{Problem \theexercise.} \textit{#1}\par\medskip
}
{%
  \end{exercisebox}
}

% =====================================================
% Solution Environment
% =====================================================
\newenvironment{solution}
{\par\noindent\textbf{Solution.}\ }
{\hfill$\square$\par}

% =====================================================
% Common Commands
% =====================================================
\newcommand{\R}{\mathbb{R}}
\newcommand{\C}{\mathbb{C}}
\newcommand{\F}{\mathbb{F}}
\newcommand{\Span}{\operatorname{span}}
\newcommand{\Null}{\operatorname{null}}
\newcommand{\Range}{\operatorname{range}}

% =====================================================
% Metadata
% =====================================================
\title{Linear Algebra Done Right \\ \large Section 1C: Subspaces}
\author{Naman Tyagi}
\date{}

% =====================================================
% Document
% =====================================================
\begin{document}
\maketitle

\begin{exercise}{}
For each of the following subsets of $\mathbb{F}^3$, determine whether it is a
subspace of $\mathbb{F}^3$.
\end{exercise}

\begin{solution}
\textbf{(a)} 
\[
\{(x_1,x_2,x_3)\in\mathbb{F}^3 : x_1+2x_2+3x_3=0\}
\]
This set is a subspace of $\mathbb{F}^3$. The zero vector satisfies the equation,
and the set is closed under addition and scalar multiplication.

\medskip

\textbf{(b)}
\[
\{(x_1,x_2,x_3)\in\mathbb{F}^3 : x_1+2x_2+3x_3=4\}
\]
This set is not a subspace of $\mathbb{F}^3$ because the zero vector does not
satisfy the defining equation.

\medskip

\textbf{(c)}
\[
\{(x_1,x_2,x_3)\in\mathbb{F}^3 : x_1x_2x_3=0\}
\]
This set is not a subspace of $\mathbb{F}^3$. Although the zero vector belongs to
the set, it is not closed under addition. For example,
\[
(1,0,1),(0,1,1)\in U,
\]
but
\[
(1,0,1)+(0,1,1)=(1,1,2)\notin U.
\]

\medskip

\textbf{(d)}
\[
\{(x_1,x_2,x_3)\in\mathbb{F}^3 : x_1=5x_3\}
\]
This set is a subspace of $\mathbb{F}^3$. The zero vector satisfies the condition,
and the set is closed under addition and scalar multiplication.
\end{solution}

\begin{exercise}{}
Verify all assertions about subspaces in Example 1.35.
\end{exercise}

\begin{solution}
\textbf{(a)}
If $b \in \mathbb{F}$, consider
\[
\{(x_1,x_2,x_3,x_4)\in\mathbb{F}^4 : x_3 = 5x_4 + b\}.
\]
The zero vector belongs to this set if and only if $0 = 5\cdot 0 + b$, which
holds if and only if $b=0$. When $b=0$, the defining condition is linear and the
set is closed under addition and scalar multiplication. Hence the set is a
subspace of $\mathbb{F}^4$ if and only if $b=0$.

\medskip

\textbf{(b)}
The set of continuous real-valued functions on $[0,1]$ contains the zero
function. The sum of two continuous functions is continuous, and any scalar
multiple of a continuous function is continuous. Therefore this set is a
subspace of $\mathbb{R}^{[0,1]}$.

\medskip

\textbf{(c)}
The set of differentiable real-valued functions on $\mathbb{R}$ contains the
zero function. It is closed under addition and scalar multiplication because
the sum and scalar multiple of differentiable functions are differentiable.
Hence it is a subspace of $\mathbb{R}^{\mathbb{R}}$.

\medskip

\textbf{(d)}
Let $b \in \mathbb{R}$ and consider the set of differentiable real-valued
functions $f$ on $(0,3)$ such that $f'(2)=b$. The zero function satisfies this
condition if and only if $b=0$. When $b=0$, the set is closed under addition and
scalar multiplication. Thus this set is a subspace of $\mathbb{R}^{(0,3)}$ if
and only if $b=0$.

\medskip

\textbf{(e)}
The set of all sequences of complex numbers with limit $0$ contains the zero
sequence. If two sequences converge to $0$, then their sum also converges to
$0$, and any scalar multiple of a sequence converging to $0$ again converges to
$0$. Therefore this set is a subspace of $\mathbb{C}^\infty$.
\end{solution}

\begin{exercise}{}
Show that the set of differentiable real-valued functions $f$ on the interval
$(-4,4)$ such that
\[
f'(-1) = 3f(2)
\]
is a subspace of $\mathbb{R}^{(-4,4)}$.
\end{exercise}

\begin{solution}
The zero function belongs to the set, since
\[
0'(-1) = 0 = 3\cdot 0(2).
\]

Let $f$ and $g$ be functions in the set. Then
\[
(f+g)'(-1) = f'(-1) + g'(-1) = 3f(2) + 3g(2) = 3(f+g)(2),
\]
so the set is closed under addition.

Let $f$ be in the set and let $a \in \mathbb{R}$. Then
\[
(af)'(-1) = a f'(-1) = a \cdot 3f(2) = 3(af)(2),
\]
so the set is closed under scalar multiplication.

Hence the given set is a subspace of $\mathbb{R}^{(-4,4)}$.

\end{solution}

\begin{exercise}{}
Suppose $b \in \mathbb{R}$. Show that the set of continuous real-valued functions
$f$ on $[0,1]$ such that
\[
\int_0^1 f = b
\]
is a subspace of $\mathbb{R}^{[0,1]}$ if and only if $b = 0$.
\end{exercise}

\begin{solution}
For the set to be a subspace, it must contain the zero vector. The zero function
satisfies
\[
\int_0^1 0 = 0,
\]
so the zero function belongs to the set if and only if $b = 0$.

If $b = 0$, the set contains the zero function. Moreover, if $f$ and $g$ are in
the set, then
\[
\int_0^1 (f+g) = \int_0^1 f + \int_0^1 g = 0,
\]
and for any $a \in \mathbb{R}$,
\[
\int_0^1 af = a \int_0^1 f = 0.
\]
Thus the set is closed under addition and scalar multiplication, and hence is a
subspace.

If $b \neq 0$, the zero function does not belong to the set, so the set cannot be
a subspace.
\end{solution}

\begin{exercise}{}
Is $\mathbb{R}^2$ a subspace of the complex vector space $\mathbb{C}^2$?
\end{exercise}

\begin{solution}
No. Although $\mathbb{R}^2$ is closed under addition and contains the zero
vector, it is not closed under scalar multiplication by complex scalars.

For example, consider the vector $(1,0) \in \mathbb{R}^2$. Multiplying by the
complex scalar $i$ gives
\[
i(1,0) = (i,0),
\]
which does not belong to $\mathbb{R}^2$. Hence $\mathbb{R}^2$ is not a subspace
of $\mathbb{C}^2$.
\end{solution}

\begin{exercise}{}
(a) Is $\{(a,b,c)\in\mathbb{R}^3 : a^3=b^3\}$ a subspace of $\mathbb{R}^3$?

(b) Is $\{(a,b,c)\in\mathbb{C}^3 : a^3=b^3\}$ a subspace of $\mathbb{C}^3$?
\end{exercise}

\begin{solution}
\textbf{(a)} Over $\mathbb{R}$, the function $x \mapsto x^3$ is injective.
Hence $a^3=b^3$ implies $a=b$. Therefore the set can be written as
\[
\{(a,a,c) : a,c \in \mathbb{R}\},
\]
which is a plane through the origin. It contains the zero vector and is closed
under addition and scalar multiplication. Thus it is a subspace of
$\mathbb{R}^3$.

\medskip

\textbf{(b)} Over $\mathbb{C}$, the implication $a^3=b^3 \Rightarrow a=b$ does not
hold, because complex numbers have nontrivial cube roots of unity. In fact,
$a^3=b^3$ if and only if $a=\omega b$ for some cube root of unity $\omega$.

The set is therefore a union of three distinct planes and is not closed under
addition. Hence it is not a subspace of $\mathbb{C}^3$.
\end{solution}

\begin{exercise}{}
Prove or give a counterexample: If $U$ is a nonempty subset of $\mathbb{R}^2$
such that $U$ is closed under addition and under taking additive inverses, then
$U$ is a subspace of $\mathbb{R}^2$.
\end{exercise}

\begin{solution}
The statement is false. Consider the set
\[
U = \{(a,b) \in \mathbb{R}^2 : a,b \in \mathbb{Z}\}.
\]

The set $U$ is nonempty and contains the zero vector $(0,0)$. It is closed under
addition because the sum of two integer vectors has integer components, and it
is closed under additive inverses because $-(a,b)=(-a,-b)$ also has integer
components.

However, $U$ is not closed under scalar multiplication by real numbers. For
example,
\[
\frac{1}{2}(1,1) = \left(\frac{1}{2}, \frac{1}{2}\right) \notin U.
\]
Hence $U$ is not a subspace of $\mathbb{R}^2$.

Therefore, closure under addition and additive inverses alone does not imply that
a set is a subspace.
\end{solution}

\begin{exercise}{}
Give an example of a nonempty subset $U$ of $\mathbb{R}^2$ such that $U$ is
closed under scalar multiplication, but $U$ is not a subspace of
$\mathbb{R}^2$.
\end{exercise}

\begin{solution}
Consider the set
\[
U = \{(x,y) \in \mathbb{R}^2 : y = x^2\}.
\]

The set $U$ is nonempty because $(0,0) \in U$.

If $(x,x^2) \in U$ and $a \in \mathbb{R}$, then
\[
a(x,x^2) = (ax, a^2 x^2) = (ax, (ax)^2),
\]
which belongs to $U$. Hence $U$ is closed under scalar multiplication.

However, $U$ is not closed under addition. For example,
\[
(1,1),(2,4) \in U,
\]
but
\[
(1,1) + (2,4) = (3,5) \notin U,
\]
since $5 \neq 3^2$.

Therefore, $U$ is not a subspace of $\mathbb{R}^2$.
\end{solution}

\begin{exercise}{}
A function $f:\mathbb{R}\to\mathbb{R}$ is called periodic if there exists a
positive number $p$ such that $f(x)=f(x+p)$ for all $x\in\mathbb{R}$. Is the set
of periodic functions from $\mathbb{R}$ to $\mathbb{R}$ a subspace of
$\mathbb{R}^{\mathbb{R}}$? Explain.
\end{exercise}

\begin{solution}
No. The set of periodic functions is not a subspace of $\mathbb{R}^{\mathbb{R}}$
because it is not closed under addition.

Define
\[
f(x)=\cos x \quad \text{and} \quad g(x)=\cos(\pi x).
\]
The function $f$ is periodic with period $2\pi$, and $g$ is periodic with
period $2$.

Suppose, for contradiction, that $f+g$ is periodic. Then there exists a
positive number $p$ such that
\[
\cos(x+p)+\cos(\pi(x+p))=\cos x+\cos(\pi x)
\quad \text{for all } x\in\mathbb{R}.
\]
This implies that $p$ is a common period of both $f$ and $g$. Hence there exist
integers $m,n$ such that
\[
p = 2\pi n \quad \text{and} \quad p = 2m.
\]
Equating these expressions gives
\[
2\pi n = 2m,
\]
and thus
\[
\pi = \frac{m}{n},
\]
which is impossible because $\pi$ is irrational.

Therefore, no such $p$ exists, and $f+g$ is not periodic. Hence the set of
periodic functions is not closed under addition and is not a subspace of
$\mathbb{R}^{\mathbb{R}}$.
\end{solution}

\begin{exercise}{}
Suppose $V_1$ and $V_2$ are subspaces of $V$. Prove that $V_1 \cap V_2$ is a
subspace of $V$.
\end{exercise}

\begin{solution}
Because $V_1$ and $V_2$ are subspaces of $V$, each contains the zero vector.
Hence $0 \in V_1 \cap V_2$.

Let $u,v \in V_1 \cap V_2$. Then $u,v \in V_1$ and $u,v \in V_2$. Since $V_1$
and $V_2$ are subspaces, they are closed under addition, so $u+v \in V_1$ and
$u+v \in V_2$. Thus $u+v \in V_1 \cap V_2$.

Let $a \in \mathbb{F}$ and let $u \in V_1 \cap V_2$. Then $u \in V_1$ and
$u \in V_2$. Because $V_1$ and $V_2$ are subspaces, they are closed under scalar
multiplication, so $au \in V_1$ and $au \in V_2$. Hence $au \in V_1 \cap V_2$.

Therefore $V_1 \cap V_2$ contains the zero vector and is closed under addition
and scalar multiplication. Thus $V_1 \cap V_2$ is a subspace of $V$.
\end{solution}

\begin{exercise}{}
Prove that the intersection of every collection of subspaces of $V$ is a
subspace of $V$.
\end{exercise}

\begin{solution}
Let $\{V_\alpha\}_{\alpha \in A}$ be a collection of subspaces of $V$, and let
\[
W = \bigcap_{\alpha \in A} V_\alpha.
\]

Because each $V_\alpha$ is a subspace of $V$, each contains the zero vector.
Hence $0 \in W$.

Let $u,v \in W$. Then $u,v \in V_\alpha$ for every $\alpha \in A$. Since each
$V_\alpha$ is a subspace, it is closed under addition, so $u+v \in V_\alpha$
for every $\alpha \in A$. Therefore $u+v \in W$.

Let $a \in \mathbb{F}$ and let $u \in W$. Then $u \in V_\alpha$ for every
$\alpha \in A$. Because each $V_\alpha$ is closed under scalar multiplication,
we have $au \in V_\alpha$ for every $\alpha \in A$. Hence $au \in W$.

Thus $W$ contains the zero vector and is closed under addition and scalar
multiplication. Therefore $W$ is a subspace of $V$.
\end{solution}

\begin{exercise}{}
Prove that the union of two subspaces of $V$ is a subspace of $V$ if and only if
one of the subspaces is contained in the other.
\end{exercise}

\begin{solution}
Let $V_1$ and $V_2$ be subspaces of $V$.

\medskip

\textbf{($\Rightarrow$)}  
Assume that $V_1 \cup V_2$ is a subspace of $V$. Suppose, for contradiction,
that neither subspace is contained in the other. Then there exists
$u \in V_1 \setminus V_2$ and $v \in V_2 \setminus V_1$.

Since $u,v \in V_1 \cup V_2$ and $V_1 \cup V_2$ is a subspace, it is closed under
addition, so $u+v \in V_1 \cup V_2$. Thus either $u+v \in V_1$ or $u+v \in V_2$.

If $u+v \in V_1$, then
\[
v = (u+v) - u \in V_1,
\]
since $V_1$ is a subspace. This contradicts $v \notin V_1$.

If $u+v \in V_2$, then
\[
u = (u+v) - v \in V_2,
\]
since $V_2$ is a subspace. This contradicts $u \notin V_2$.

Both cases lead to contradictions. Hence one of the subspaces must be contained
in the other.

\medskip

\textbf{($\Leftarrow$)}  
If $V_1 \subseteq V_2$, then
\[
V_1 \cup V_2 = V_2,
\]
which is a subspace of $V$. Similarly, if $V_2 \subseteq V_1$, then
$V_1 \cup V_2 = V_1$, which is also a subspace.

\medskip

Therefore, the union of two subspaces of $V$ is a subspace of $V$ if and only if
one of the subspaces is contained in the other.
\end{solution}

\begin{exercise}{}
Prove that the union of three subspaces of $V$ is a subspace of $V$ if and only if
one of the subspaces contains the other two.
\end{exercise}

\begin{solution}
Let $V_1$, $V_2$, and $V_3$ be subspaces of $V$.

\medskip

\textbf{($\Rightarrow$)}  
Assume that $V_1 \cup V_2 \cup V_3$ is a subspace of $V$. Suppose, for
contradiction, that none of the subspaces contains the other two. Then there
exist
\[
u \in V_1 \setminus (V_2 \cup V_3)
\quad \text{and} \quad
v \in V_2 \setminus (V_1 \cup V_3).
\]
Both $u$ and $v$ belong to $V_1 \cup V_2 \cup V_3$. Since this union is a
subspace, it is closed under addition, so
\[
u+v \in V_1 \cup V_2 \cup V_3.
\]

If $u+v \in V_1$, then
\[
v = (u+v) - u \in V_1,
\]
a contradiction. If $u+v \in V_2$, then
\[
u = (u+v) - v \in V_2,
\]
also a contradiction.

Thus $u+v \in V_3$. Because $V_3$ is a subspace, it is closed under additive
inverses, so
\[
u = (u+v) - v \in V_3,
\]
which contradicts $u \notin V_3$.

Hence our assumption was false, and one of the subspaces must contain the other
two.

\medskip

\textbf{($\Leftarrow$)}  
If one of the subspaces contains the other two, say
\[
V_1 \supseteq V_2 \cup V_3,
\]
then
\[
V_1 \cup V_2 \cup V_3 = V_1,
\]
which is a subspace of $V$.

\medskip

Therefore, the union of three subspaces of $V$ is a subspace of $V$ if and only
if one of the subspaces contains the other two.
\end{solution}

\begin{exercise}{}
Suppose
\[
U = \{(x,-x,2x)\in\mathbb{F}^3 : x\in\mathbb{F}\}
\quad \text{and} \quad
W = \{(x,x,2x)\in\mathbb{F}^3 : x\in\mathbb{F}\}.
\]
Describe $U+W$ using symbols, and also give a description of $U+W$ that uses
no symbols.
\end{exercise}

\begin{solution}
An element of $U+W$ has the form
\[
(x,-x,2x) + (y,y,2y) = (x+y,\,-x+y,\,2(x+y)),
\]
where $x,y \in \mathbb{F}$. Hence
\[
U+W = \{(a,b,2a) \in \mathbb{F}^3 : a,b \in \mathbb{F}\}.
\]

In words, $U+W$ is the set of all vectors in $\mathbb{F}^3$ whose third
coordinate is twice the first coordinate.
\end{solution}

\begin{exercise}{}
Suppose $U$ is a subspace of $V$. What is $U + U$?
\end{exercise}

\begin{solution}
We claim that $U + U = U$.

First, let $u_1, u_2 \in U$. Since $U$ is a subspace, it is closed under
addition, so $u_1 + u_2 \in U$. Hence $U + U \subseteq U$.

Conversely, let $u \in U$. Because $0 \in U$, we may write
\[
u = u + 0,
\]
with both $u$ and $0$ belonging to $U$. Thus $u \in U + U$, and so
$U \subseteq U + U$.

Therefore $U + U = U$.
\end{solution}

\begin{exercise}{}
Is the operation of addition on the subspaces of $V$ commutative? In other
words, if $U$ and $W$ are subspaces of $V$, is $U + W = W + U$?
\end{exercise}

\begin{solution}
Yes. Let $x \in U + W$. Then $x = u + w$ for some $u \in U$ and $w \in W$.
Because vector addition in $V$ is commutative,
\[
x = u + w = w + u.
\]
Since $w \in W$ and $u \in U$, this shows that $x \in W + U$. Hence
$U + W \subseteq W + U$.

By the same argument with the roles of $U$ and $W$ reversed,
$W + U \subseteq U + W$. Therefore,
\[
U + W = W + U.
\]
\end{solution}

\begin{exercise}{}
Is the operation of addition on the subspaces of $V$ associative? In other
words, if $V_1, V_2, V_3$ are subspaces of $V$, is
\[
(V_1 + V_2) + V_3 = V_1 + (V_2 + V_3)?
\]
\end{exercise}

\begin{solution}
Yes. Let $x \in (V_1 + V_2) + V_3$. Then there exist vectors
$u \in V_1$, $v \in V_2$, and $w \in V_3$ such that
\[
x = (u + v) + w.
\]
By associativity of vector addition in $V$,
\[
x = u + (v + w).
\]
Since $v + w \in V_2 + V_3$, it follows that $x \in V_1 + (V_2 + V_3)$. Hence
\[
(V_1 + V_2) + V_3 \subseteq V_1 + (V_2 + V_3).
\]

Conversely, let $x \in V_1 + (V_2 + V_3)$. Then there exist vectors
$u \in V_1$, $v \in V_2$, and $w \in V_3$ such that
\[
x = u + (v + w).
\]
Again using associativity of vector addition,
\[
x = (u + v) + w.
\]
Since $u + v \in V_1 + V_2$, we have $x \in (V_1 + V_2) + V_3$. Thus
\[
V_1 + (V_2 + V_3) \subseteq (V_1 + V_2) + V_3.
\]

Therefore,
\[
(V_1 + V_2) + V_3 = V_1 + (V_2 + V_3).
\]
\end{solution}

\begin{exercise}{}
Does the operation of addition on the subspaces of $V$ have an additive
identity? Which subspaces have additive inverses?
\end{exercise}

\begin{solution}
Yes. The additive identity is the zero subspace $\{0\}$. For any subspace
$U \subseteq V$,
\[
U + \{0\} = U,
\]
because every element of $U + \{0\}$ has the form $u + 0 = u$ with $u \in U$.

Next, we determine which subspaces have additive inverses. A subspace $W$ would
be an additive inverse of $U$ if
\[
U + W = \{0\}.
\]
This is possible only if $U = \{0\}$. Indeed, if $U \neq \{0\}$, then $U$
contains a nonzero vector, and hence $U + W$ must contain nonzero vectors for
any subspace $W$, so it cannot equal $\{0\}$.

Therefore, the only subspace that has an additive inverse is the zero subspace
itself.
\end{solution}

\begin{exercise}{}
Prove or give a counterexample: If $V_1$, $V_2$, and $U$ are subspaces of $V$
such that
\[
V_1 + U = V_2 + U,
\]
then $V_1 = V_2$.
\end{exercise}

\begin{solution}
The statement is false.

Let
\[
V = \mathbb{R}^3,
\quad
U = \operatorname{span}\{(0,0,1)\},
\quad
V_1 = \operatorname{span}\{(1,0,0)\},
\quad
V_2 = \operatorname{span}\{(1,0,0),(0,0,1)\}.
\]

Clearly, $V_1 \neq V_2$.

However,
\[
V_1 + U
= \operatorname{span}\{(1,0,0)\} + \operatorname{span}\{(0,0,1)\}
= \operatorname{span}\{(1,0,0),(0,0,1)\},
\]
and
\[
V_2 + U
= \operatorname{span}\{(1,0,0),(0,0,1)\} + \operatorname{span}\{(0,0,1)\}
= \operatorname{span}\{(1,0,0),(0,0,1)\}.
\]

Thus,
\[
V_1 + U = V_2 + U
\quad \text{but} \quad
V_1 \neq V_2.
\]

Therefore, subspace addition does not satisfy cancellation.
\end{solution}

\begin{exercise}{}
Suppose
\[
U = \{(x,x,y,y)\in\mathbb{F}^4 : x,y \in \mathbb{F}\}.
\]
Find a subspace $W$ of $\mathbb{F}^4$ such that
\[
\mathbb{F}^4 = U \oplus W.
\]
\end{exercise}

\begin{solution}
Let
\[
W = \{(a,-a,b,-b)\in\mathbb{F}^4 : a,b \in \mathbb{F}\}.
\]

We show that $\mathbb{F}^4 = U \oplus W$.

First, we show that $U + W = \mathbb{F}^4$. Let
$(u_1,u_2,u_3,u_4) \in \mathbb{F}^4$. Then
\[
(u_1,u_2,u_3,u_4)
=
\left(\tfrac{u_1+u_2}{2},\tfrac{u_1+u_2}{2},
\tfrac{u_3+u_4}{2},\tfrac{u_3+u_4}{2}\right)
+
\left(\tfrac{u_1-u_2}{2},-\tfrac{u_1-u_2}{2},
\tfrac{u_3-u_4}{2},-\tfrac{u_3-u_4}{2}\right),
\]
where the first vector belongs to $U$ and the second belongs to $W$. Hence
$U+W=\mathbb{F}^4$.

Next, we show that $U \cap W = \{0\}$. If
\[
(x,x,y,y) = (a,-a,b,-b),
\]
then $x=a=-a$ and $y=b=-b$, which implies $x=y=0$. Therefore
$U \cap W = \{0\}$.

Thus $\mathbb{F}^4 = U \oplus W$.
\end{solution}

\begin{exercise}{}
Suppose
\[
U=\{(x,y,x+y,x-y,2x)\in\mathbb{F}^5 : x,y\in\mathbb{F}\}.
\]
Find a subspace $W$ of $\mathbb{F}^5$ such that
\[
\mathbb{F}^5 = U \oplus W.
\]
\end{exercise}

\begin{solution}
We first write $U$ as a span:
\[
(x,y,x+y,x-y,2x)
= x(1,0,1,1,2) + y(0,1,1,-1,0).
\]
Hence
\[
U=\operatorname{span}\{(1,0,1,1,2),(0,1,1,-1,0)\},
\]
and therefore $\dim U=2$.

Since $\dim\mathbb{F}^5=5$, a complementary subspace $W$ must have dimension
$3$. Define
\[
W=\operatorname{span}\{(0,0,1,0,0),(0,0,0,1,0),(0,0,0,0,1)\}.
\]

We show that $\mathbb{F}^5=U\oplus W$.

First, the five vectors
\[
(1,0,1,1,2),\ (0,1,1,-1,0),\ (0,0,1,0,0),\ (0,0,0,1,0),\ (0,0,0,0,1)
\]
are linearly independent and thus span $\mathbb{F}^5$. Hence $U+W=\mathbb{F}^5$.

Next, suppose a vector lies in $U\cap W$. From $W$, its first two coordinates
are zero, while from $U$ it has the form $(x,y,x+y,x-y,2x)$. This forces
$x=0$ and $y=0$, so the vector is the zero vector. Thus
\[
U\cap W=\{0\}.
\]

Therefore,
\[
\mathbb{F}^5 = U \oplus W.
\]
\end{solution}

\begin{exercise}{}
Suppose
\[
U=\{(x,y,x+y,x-y,2x)\in\mathbb{F}^5 : x,y\in\mathbb{F}\}.
\]
Find three subspaces $W_1, W_2, W_3$ of $\mathbb{F}^5$, none of which equals
$\{0\}$, such that
\[
\mathbb{F}^5 = U \oplus (W_1 \oplus W_2 \oplus W_3).
\]
\end{exercise}

\begin{solution}
We first write $U$ as a span:
\[
(x,y,x+y,x-y,2x)
= x(1,0,1,1,2) + y(0,1,1,-1,0).
\]
Hence
\[
U = \operatorname{span}\{(1,0,1,1,2),(0,1,1,-1,0)\},
\]
so $\dim U = 2$.

Define
\[
W_1=\operatorname{span}\{(0,0,1,0,0)\}, \quad
W_2=\operatorname{span}\{(0,0,0,1,0)\}, \quad
W_3=\operatorname{span}\{(0,0,0,0,1)\}.
\]

Each $W_i$ is a nonzero subspace of $\mathbb{F}^5$. The vectors spanning
$U, W_1, W_2,$ and $W_3$ are linearly independent, and together they span
$\mathbb{F}^5$. Therefore,
\[
\mathbb{F}^5 = U \oplus W_1 \oplus W_2 \oplus W_3.
\]
\end{solution}

\begin{exercise}{}
Prove or give a counterexample: If $V_1$, $V_2$, and $U$ are subspaces of $V$
such that
\[
V = V_1 \oplus U \quad \text{and} \quad V = V_2 \oplus U,
\]
then $V_1 = V_2$.
\end{exercise}

\begin{solution}
The statement is false.

Let $V=\mathbb{R}^2$ and let
\[
U=\operatorname{span}\{(1,0)\}.
\]
Define
\[
V_1=\operatorname{span}\{(0,1)\}
\quad \text{and} \quad
V_2=\operatorname{span}\{(1,1)\}.
\]

Then $U \cap V_1 = \{0\}$ and $U \cap V_2 = \{0\}$. Moreover,
\[
V_1 + U = \mathbb{R}^2
\quad \text{and} \quad
V_2 + U = \mathbb{R}^2.
\]
Hence
\[
V=\mathbb{R}^2 = V_1 \oplus U = V_2 \oplus U.
\]

However, $V_1 \neq V_2$. Therefore, even if two direct sum decompositions share
the same subspace $U$, the complementary subspace need not be unique.
\end{solution}

\begin{exercise}{}
A function $f:\mathbb{R}\to\mathbb{R}$ is called even if $f(-x)=f(x)$ for all
$x\in\mathbb{R}$, and odd if $f(-x)=-f(x)$ for all $x\in\mathbb{R}$. Let $V_e$
denote the set of real-valued even functions on $\mathbb{R}$ and let $V_o$
denote the set of real-valued odd functions on $\mathbb{R}$. Show that
\[
\mathbb{R}^{\mathbb{R}} = V_e \oplus V_o.
\]
\end{exercise}

\begin{solution}
First, we show that every function $f \in \mathbb{R}^{\mathbb{R}}$ can be written
as a sum of an even function and an odd function. Define
\[
f_e(x) = \frac{f(x)+f(-x)}{2}
\quad \text{and} \quad
f_o(x) = \frac{f(x)-f(-x)}{2}.
\]
Then $f_e$ is even and $f_o$ is odd. Moreover,
\[
f(x) = f_e(x) + f_o(x)
\quad \text{for all } x \in \mathbb{R}.
\]
Hence
\[
\mathbb{R}^{\mathbb{R}} = V_e + V_o.
\]

Next, we show that the sum is direct. Suppose $f \in V_e \cap V_o$. Then $f$ is
both even and odd, so for all $x$,
\[
f(x) = f(-x) = -f(x),
\]
which implies $f(x)=0$ for all $x$. Thus
\[
V_e \cap V_o = \{0\}.
\]

Therefore,
\[
\mathbb{R}^{\mathbb{R}} = V_e \oplus V_o.
\]
\end{solution}

\end{document}
