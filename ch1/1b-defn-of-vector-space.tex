\documentclass[11pt]{article}

% =====================================================
% Layout & Typography
% =====================================================
\usepackage[margin=1in]{geometry}
\usepackage{setspace}
\setstretch{1.15}

\usepackage[T1]{fontenc}
\usepackage{lmodern}
\usepackage{microtype}

% =====================================================
% Math
% =====================================================
\usepackage{amsmath, amssymb}
\usepackage{mathtools}

% =====================================================
% Lists & Links
% =====================================================
\usepackage{enumitem}
\usepackage{hyperref}
\hypersetup{
  colorlinks=true,
  linkcolor=blue,
  urlcolor=blue
}

% =====================================================
% Framed Exercise Environment (LEGAL)
% =====================================================
\usepackage{mdframed}

\newcounter{exercise}

\newmdenv[
  linewidth=0.6pt,
  skipabove=1.2em,
  skipbelow=1.2em,
  innerleftmargin=1em,
  innerrightmargin=1em,
  innertopmargin=0.8em,
  innerbottommargin=0.8em
]{exercisebox}

\newenvironment{exercise}[1]
{%
  \refstepcounter{exercise}
  \begin{exercisebox}
  \noindent\textbf{Problem \theexercise.} \textit{#1}\par\medskip
}
{%
  \end{exercisebox}
}

% =====================================================
% Solution Environment
% =====================================================
\newenvironment{solution}
{\par\noindent\textbf{Solution.}\ }
{\hfill$\square$\par}

% =====================================================
% Common Commands
% =====================================================
\newcommand{\R}{\mathbb{R}}
\newcommand{\C}{\mathbb{C}}
\newcommand{\F}{\mathbb{F}}
\newcommand{\Span}{\operatorname{span}}
\newcommand{\Null}{\operatorname{null}}
\newcommand{\Range}{\operatorname{range}}

% =====================================================
% Metadata
% =====================================================
\title{Linear Algebra Done Right \\ \large Section 1B: Definition of Vector Space}
\author{Naman Tyagi}
\date{}

% =====================================================
% Document
% =====================================================
\begin{document}
\maketitle

\begin{exercise}{}
Prove that $-(-v) = v$ for every $v \in V$.
\end{exercise}

\begin{solution}
Let $v \in V$ be arbitrary. Write
\[
v = (x_1, x_2, \dots).
\]
Then
\[
-v = (-x_1, -x_2, \dots).
\]
Taking the negative again,
\[
-(-v) = (-(-x_1), -(-x_2), \dots) = (x_1, x_2, \dots) = v.
\]
\end{solution}

\begin{exercise}{}
Suppose $a \in \mathbb{F}$, $v \in V$, and $av = 0$. Prove that $a = 0$ or $v = 0$.
\end{exercise}

\begin{solution}
Let
\[
v = (x_1, x_2, \dots).
\]
Then
\[
av = (ax_1, ax_2, \dots).
\]

We are given that
\[
av = (0, 0, \dots).
\]
Hence,
\[
(ax_1, ax_2, \dots) = (0, 0, \dots).
\]

Equating corresponding components, we obtain
\[
ax_1 = 0, \; ax_2 = 0, \; \dots
\]

Thus, either $a = 0$ or $x_1 = 0$.  
If $a \neq 0$, then
\[
x_1 = x_2 = \cdots = 0,
\]
which implies $v = 0$.

Therefore, $a = 0$ or $v = 0$.
\end{solution}

\begin{exercise}{}
Suppose $a \in \mathbb{F}$, $v \in V$, and $av = 0$. Prove that $a = 0$ or $v = 0$.
\end{exercise}

\begin{solution}
Let
\[
v = (x_1, x_2, \dots).
\]
Then
\[
av = (ax_1, ax_2, \dots).
\]

We are given that
\[
av = (0, 0, \dots).
\]
Hence,
\[
(ax_1, ax_2, \dots) = (0, 0, \dots).
\]

Equating corresponding components, we obtain
\[
ax_1 = 0, \; ax_2 = 0, \; \dots
\]

Thus, either $a = 0$ or $x_1 = 0$.  
If $a \neq 0$, then
\[
x_1 = x_2 = \cdots = 0,
\]
which implies $v = 0$.

Therefore, $a = 0$ or $v = 0$.
\end{solution}

\begin{exercise}{}
Suppose $v, w \in V$. Explain why there exists a unique $x \in V$ such that
\[
v + 3x = w.
\]
\end{exercise}

\begin{solution}
We are given
\[
v + 3x = w.
\]

Writing components,
\[
(v_1, v_2, \dots) + (3x_1, 3x_2, \dots) = (w_1, w_2, \dots).
\]

Equating components,
\[
v_1 + 3x_1 = w_1.
\]

Solving for $x_1$,
\[
x_1 = \frac{w_1 - v_1}{3}
= \frac{1}{3}w_1 + \frac{1}{3}(-v_1).
\]

Extending this to all components,
\[
x = \left( \frac{w_1 - v_1}{3}, \frac{w_2 - v_2}{3}, \dots \right).
\]

Since vector addition and scalar multiplication are closed in vector spaces, we have $x \in V$.
\end{solution}

\begin{solution}[Alternate solution]
Starting from
\[
v + 3x = w,
\]
subtract $v$ from both sides to obtain
\[
3x = w - v.
\]

Since $3 \neq 0$ in the field $\mathbb{F}$, scalar multiplication by $3$ is invertible. Thus,
\[
x = \frac{1}{3}(w - v).
\]

This element exists in $V$ by closure under scalar multiplication, and it is unique because scalar multiplication by a nonzero scalar is injective.
\end{solution}

\begin{exercise}{}
The empty set is not a vector space. The empty set fails to satisfy only one
of the requirements listed in the definition of a vector space. Which one?
\end{exercise}

\begin{solution}
All axioms of a vector space that are stated in the form
\[
\text{for all } v, w \in V, \ \dots
\]
are vacuously true when $V = \varnothing$.

However, the axiom requiring the existence of an additive identity states that
there exists a vector $0 \in V$ such that
\[
v + 0 = v \quad \text{for all } v \in V.
\]

Since the empty set contains no elements, it cannot contain a zero vector.
Therefore, the empty set fails the axiom requiring the existence of an additive
identity.
\end{solution}

\newpage

\begin{exercise}{}
Show that in the definition of a vector space (1.20), the additive inverse
condition can be replaced with the condition that
\[
0v = 0 \quad \text{for all } v \in V,
\]
where the $0$ on the left is the scalar $0$ and the $0$ on the right is the
additive identity of $V$.
\end{exercise}

\begin{solution}
We show that the collection of objects satisfying the vector space axioms is
unchanged if the additive inverse axiom is replaced by the condition
\[
0v = 0 \quad \text{for all } v \in V.
\]

First, assume the usual vector space axioms, including the existence of
additive inverses. Then for any $v \in V$,
\[
0v = (0 + 0)v = 0v + 0v.
\]
Adding the additive inverse of $0v$ to both sides gives
\[
0 = 0v,
\]
so the condition $0v = 0$ holds.

Conversely, assume all vector space axioms except the additive inverse axiom,
and instead assume that
\[
0v = 0 \quad \text{for all } v \in V.
\]
Let $v \in V$. Using distributivity,
\[
v + (-1)v = (1 + (-1))v = 0v = 0.
\]
Thus $(-1)v$ is an additive inverse of $v$.

Therefore, the existence of additive inverses follows from the condition
$0v = 0$, and the two definitions are equivalent.
\end{solution}

\newpage

\begin{exercise}{}
Let $\infty$ and $-\infty$ denote two distinct objects, neither of which is in
$\mathbb{R}$. Define addition and scalar multiplication on
$\mathbb{R} \cup \{\infty, -\infty\}$ as follows: addition and scalar
multiplication of real numbers are as usual, and for $t \in \mathbb{R}$ define
\[
t\infty =
\begin{cases}
-\infty & \text{if } t < 0, \\
0 & \text{if } t = 0, \\
\infty & \text{if } t > 0,
\end{cases}
\qquad
t(-\infty) =
\begin{cases}
\infty & \text{if } t < 0, \\
0 & \text{if } t = 0, \\
-\infty & \text{if } t > 0,
\end{cases}
\]
and
\[
t + \infty = \infty + t = \infty + \infty = \infty,
\]
\[
t + (-\infty) = (-\infty) + t = (-\infty) + (-\infty) = -\infty,
\]
\[
\infty + (-\infty) = (-\infty) + \infty = 0.
\]
With these operations, is $\mathbb{R} \cup \{\infty, -\infty\}$ a vector space
over $\mathbb{R}$? Explain.
\end{exercise}

\begin{solution}
No. The set $\mathbb{R} \cup \{\infty, -\infty\}$ is not a vector space over
$\mathbb{R}$ because addition is not associative.

Indeed,
\[
(\infty + \infty) + (-\infty) = \infty + (-\infty) = 0,
\]
whereas
\[
\infty + (\infty + (-\infty)) = \infty + 0 = \infty.
\]
Since
\[
(\infty + \infty) + (-\infty) \neq \infty + (\infty + (-\infty)),
\]
addition is not associative. Therefore, $\mathbb{R} \cup \{\infty, -\infty\}$
does not satisfy the axioms of a vector space.
\end{solution}

\begin{exercise}{}
Suppose $S$ is a nonempty set. Let $V^S$ denote the set of functions from $S$ to
$V$. Define a natural addition and scalar multiplication on $V^S$, and show
that $V^S$ is a vector space with these definitions.
\end{exercise}

\begin{solution}
Let $f, g \in V^S$ and let $a \in \mathbb{F}$.

Define addition in $V^S$ by
\[
(f + g)(s) = f(s) + g(s) \quad \text{for all } s \in S,
\]
and define scalar multiplication by
\[
(af)(s) = a f(s) \quad \text{for all } s \in S.
\]

We verify the vector space axioms.

Since $V$ is closed under addition and scalar multiplication, the functions
$f+g$ and $af$ map $S$ into $V$, so $V^S$ is closed under these operations.

Addition in $V^S$ is commutative and associative because addition in $V$ is
commutative and associative:
\[
(f+g)(s) = f(s) + g(s) = g(s) + f(s) = (g+f)(s),
\]
and
\[
((f+g)+h)(s) = (f(s)+g(s))+h(s) = f(s)+(g(s)+h(s)) = (f+(g+h))(s).
\]

Define the zero vector $0 \in V^S$ by
\[
0(s) = 0_V \quad \text{for all } s \in S,
\]
where $0_V$ is the zero vector in $V$. Then for all $f \in V^S$,
\[
(f+0)(s) = f(s) + 0_V = f(s),
\]
so $0$ is the additive identity.

For each $f \in V^S$, define $-f \in V^S$ by
\[
(-f)(s) = -f(s).
\]
Then
\[
(f + (-f))(s) = f(s) + (-f(s)) = 0_V,
\]
so every element has an additive inverse.

The distributive and scalar multiplication axioms follow directly from the
corresponding axioms in $V$, since all operations are defined pointwise.

Therefore, $V^S$ is a vector space.
\end{solution}

\begin{exercise}{}
Suppose $V$ is a real vector space. The complexification of $V$, denoted by
$V_{\mathbb C}$, equals $V \times V$. An element of $V_{\mathbb C}$ is written
as $u + iv$, where $u, v \in V$. Addition on $V_{\mathbb C}$ is defined by
\[
(u_1 + iv_1) + (u_2 + iv_2) = (u_1 + u_2) + i(v_1 + v_2),
\]
and complex scalar multiplication is defined by
\[
(a+bi)(u+iv) = (au - bv) + i(av + bu),
\]
for all $a,b \in \mathbb{R}$ and all $u,v \in V$.

Prove that with these definitions, $V_{\mathbb C}$ is a complex vector space.
\end{exercise}

\begin{solution}
We verify the vector space axioms.

First, closure under addition and scalar multiplication holds because $V$ is
closed under addition and real scalar multiplication.

Addition in $V_{\mathbb C}$ is commutative and associative since addition in
$V$ is commutative and associative:
\[
(u_1 + iv_1) + (u_2 + iv_2) = (u_2 + iv_2) + (u_1 + iv_1),
\]
and
\[
((u_1 + iv_1) + (u_2 + iv_2)) + (u_3 + iv_3)
= (u_1 + iv_1) + ((u_2 + iv_2) + (u_3 + iv_3)).
\]

The additive identity is $0 + i0$, and the additive inverse of $u + iv$ is
$(-u) + i(-v)$.

Next, we verify the scalar multiplication axioms. Let
$\alpha = a+bi$ and $\beta = c+di$ be complex scalars, and let $u+iv \in
V_{\mathbb C}$.

Associativity of scalar multiplication:
\[
\alpha(\beta(u+iv)) = (a+bi)((cu-dv)+i(cv+du))
= ((ac-bd)u - (ad+bc)v) + i((ac-bd)v + (ad+bc)u),
\]
which equals
\[
(\alpha\beta)(u+iv).
\]

Identity:
\[
1(u+iv) = (1+0i)(u+iv) = u+iv.
\]

Distributivity over vector addition:
\[
\alpha((u_1+iv_1)+(u_2+iv_2)) = \alpha((u_1+u_2)+i(v_1+v_2))
= \alpha(u_1+iv_1) + \alpha(u_2+iv_2).
\]

Distributivity over scalar addition:
\[
(\alpha+\beta)(u+iv) = \alpha(u+iv) + \beta(u+iv).
\]

Thus all vector space axioms are satisfied, and $V_{\mathbb C}$ is a complex
vector space.

Identifying $u \in V$ with $u + i0$, we may view $V$ as a real subspace of
$V_{\mathbb C}$, and this construction generalizes the passage from
$\mathbb{R}^n$ to $\mathbb{C}^n$.
\end{solution}


\end{document}
